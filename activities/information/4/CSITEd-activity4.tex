\documentclass[a4paper,11pt,final]{article}

% Packages
\usepackage[french]{babel}
\usepackage[latin1]{inputenc}
\usepackage[T1]{fontenc}
\usepackage{graphicx}
\usepackage{color}
\usepackage{fancyhdr}
\usepackage{array}
\usepackage{amsmath,amssymb}
\usepackage{tikz,pgf}
\usepackage{hyperref}
\usepackage{microtype}

% Longueurs
\setlength\textheight{23cm}
\setlength\textwidth{17cm}
\setlength\oddsidemargin{-0.5cm}
\setlength\topmargin{-15mm}
\setlength\headheight{20mm}
\setlength\parindent{0.0cm}
\setlength\parskip{0.5cm}

% Style
\urlstyle{sf}

\makeatletter
\renewcommand\section{\vspace{-4mm}\@startsection {section}{1}{\z@}%
   {-3.5ex \@plus -1ex \@minus -.2ex}%
   {0.3ex \@plus.2ex}%
   {\hspace{-2mm}\noindent\normalfont\Large\bfseries}}
\makeatother

% Ent�te et pied de page
\pagestyle{fancy}
\renewcommand{\headrulewidth}{0pt}
\lhead{\includegraphics[width=2.5cm]{../../commonfiles/csited-logo.pdf}}
\rhead{\sl \#4 Cachons l'information sur un rouleau ! \\�S�bastien Comb�fis}
\lfoot{\color[gray]{0.5} \footnotesize Information / Cryptographie}
\rfoot{\vfill\sf\thepage}
\cfoot{}

% D�but du document
\begin{document}

% - - - - - - - - - - - - - - - - - - - - - - - - - - - - - - - - - - - - - - - - - - - - - - - - - - - - - -

\begin{center}{\LARGE \emph{\#4 Cachons l'information sur un rouleau !}}\end{center}

% - - - - - - - - - - - - - - - - - - - - - - - - - - - - - - - - - - - - - - - - - - - - - - - - - - - - - -

Certaines informations sont sensibles et doivent �tre cach�es afin de ne pouvoir �tre lues que par l'exp�diteur et le destinataire du message. Plusieurs techniques sont utilisables pour rendre une information illisible par tout individu malhonn�te qui volerait un message qui ne lui est pas destin�.

Dans cette activit�, on va comprendre les principes de bases de la cryptographie, une branche � la fronti�re entre l'informatique et les math�matiques. On y d�couvre notamment la notion de message clair et chiffr� et de cl� de chiffrement.

\begin{center}
	\scalebox{0.8}{\begin{tikzpicture}[font=\sf]
		\draw (0,0) ellipse[x radius=3mm,y radius=5mm];
		\draw (0,0.5) -- (4,1);
		\draw (0,-0.5) -- (4,0);
		\draw (4,0) arc[x radius=3mm,y radius=5mm,start angle=-90,end angle=90];
		%%%
		\node at (0.75,0.1) {H};
		\node at (1.5,0.2) {E};
		\node at (2.25,0.3) {L};
		\node at (3,0.4) {L};
		\node at (3.75,0.5) {O};
	\end{tikzpicture}}
\end{center}

% - - - - - - - - - - - - - - - - - - - - - - - - - - - - - - - - - - - - - - - - - - - - - - - - - - - - - -

\section{Caract�ristiques}

{\renewcommand{\arraystretch}{1.5}
\begin{tabular}{>{\bf}ll}
	Tranche d'�ge & Primaire 3--6 / Secondaire 1--6 / Adultes \\
	Temps minimal requis & 10 minutes \\
	Mode de travail & Individuel \\
	Mat�riel & -- \\
	Ressources & Bandelettes de papier et cylindres de diam�tres diff�rents \\
	Version papier & Oui \\
	Ordinateur & Non \\
	Th�mes & Information, Cryptographie
\end{tabular}}

% - - - - - - - - - - - - - - - - - - - - - - - - - - - - - - - - - - - - - - - - - - - - - - - - - - - - - -

\section{D�roulement}

\end{document}