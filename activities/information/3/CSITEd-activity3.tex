\documentclass[a4paper,11pt,final]{article}

% Packages
\usepackage[french]{babel}
\usepackage[latin1]{inputenc}
\usepackage[T1]{fontenc}
\usepackage{graphicx}
\usepackage{color}
\usepackage{fancyhdr}
\usepackage{array}
\usepackage{amsmath,amssymb}
\usepackage{tikz,pgf}
\usepackage{hyperref}
\usepackage{microtype}

% Longueurs
\setlength\textheight{23cm}
\setlength\textwidth{17cm}
\setlength\oddsidemargin{-0.5cm}
\setlength\topmargin{-15mm}
\setlength\headheight{20mm}
\setlength\parindent{0.0cm}
\setlength\parskip{0.5cm}

% Style
\urlstyle{sf}

\makeatletter
\renewcommand\section{\vspace{-4mm}\@startsection {section}{1}{\z@}%
   {-3.5ex \@plus -1ex \@minus -.2ex}%
   {0.3ex \@plus.2ex}%
   {\hspace{-2mm}\noindent\normalfont\Large\bfseries}}
\makeatother

% Ent�te et pied de page
\pagestyle{fancy}
\renewcommand{\headrulewidth}{0pt}
\lhead{\includegraphics[width=2.5cm]{../../commonfiles/csited-logo.pdf}}
\rhead{\sl \#3 O� est l'erreur de transmission ? \\�S�bastien Comb�fis}
\lfoot{\color[gray]{0.5} \footnotesize Information / Code correcteur d'erreur}
\rfoot{\vfill\sf\thepage}
\cfoot{}

% D�but du document
\begin{document}

% - - - - - - - - - - - - - - - - - - - - - - - - - - - - - - - - - - - - - - - - - - - - - - - - - - - - - -

\begin{center}{\LARGE \emph{\#3 O� est l'erreur de transmission ?}}\end{center}

% - - - - - - - - - - - - - - - - - - - - - - - - - - - - - - - - - - - - - - - - - - - - - - - - - - - - - -

Lorsqu'on stocke ou transf�re des informations en binaire, il se peut que des erreurs se produisent. Certains $1$ deviennent des $0$ et inversement. Afin de pouvoir pallier ces probl�mes, il faut am�liorer les techniques de stockage et de transmission, mais ce n'est pas suffisant. Une autre technique consiste � ajouter des informations additionnelles, redondantes, qui permettront de d�tecter et/ou corriger les �ventuelles erreurs.

Dans cette activit�, il s'agira de comprendre comment on peut utiliser un code correcteur d'erreur simple permettant de corriger au maximum une erreur. Ce code pr�sente l'information que l'on souhaite stocker/transmettre sous forme d'un tableau et se base sur le principe de la parit�.

\begin{center}
	\scalebox{0.8}{\begin{tikzpicture}
		\fill[black!50] (1,0) rectangle (1.8,0.8);
		\fill[black!50] (0,1.5) rectangle (0.8,2.3);
		\fill[black!50] (3,1.5) rectangle (3.8,2.3);
		\fill[black!50] (0,2.5) rectangle (0.8,3.3);
		\fill[black!50] (2,2.5) rectangle (2.8,3.3);
		\fill[black!50] (1,3.5) rectangle (1.8,4.3);
		\fill[black!50] (4.5,3.5) rectangle (5.3,4.3);
		\fill[black!50] (2,4.5) rectangle (2.8,5.3);
		\fill[black!50] (3,4.5) rectangle (3.8,5.3);
		%%%
		\foreach \y in {0,1.5,2.5,...,4.5}{
			\foreach \x in {0,1,...,3,4.5}{
				\draw (\x,\y) rectangle (\x+0.8,\y+0.8);
			}
		}
		\filldraw[white] (4.4,-0.2) rectangle (6.4,0.9);
	\end{tikzpicture}}
\end{center}

% - - - - - - - - - - - - - - - - - - - - - - - - - - - - - - - - - - - - - - - - - - - - - - - - - - - - - -

\section{Caract�ristiques}

{\renewcommand{\arraystretch}{1.5}
\begin{tabular}{>{\bf}ll}
	Tranche d'�ge & Primaire 5--6 / Secondaire 1--6 / Adultes\\
	Temps minimal requis & 15 minutes \\
	Mode de travail & Individuel \\
	Mat�riel & -- \\
	Ressources & 24 cartes \\
	Version papier & Oui \\
	Ordinateur & Non \\
	Th�mes & Information, Code correcteur d'erreur, Parit�
\end{tabular}}

% - - - - - - - - - - - - - - - - - - - - - - - - - - - - - - - - - - - - - - - - - - - - - - - - - - - - - -

\section{D�roulement}

\end{document}